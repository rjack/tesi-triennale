\documentclass[12pt,a4paper,openright,twoside]{book}
\usepackage[italian]{babel}
\usepackage[utf8]{inputenc}
\usepackage{fancyhdr}
\usepackage{indentfirst}
\usepackage{graphicx}

\pagestyle{fancy}

\oddsidemargin=30pt
\evensidemargin=20pt

% Sillabazione
\hyphenation{}

% Intestazione e piè di pagina
\pagestyle{fancy}\addtolength{\headwidth}{20pt}
\renewcommand{\chaptermark}[1]{\markboth{\thechapter.\ #1}{}}
\renewcommand{\sectionmark}[1]{\markright{\thesection \ #1}{}}
\rhead[\fancyplain{}{\bfseries\leftmark}]{\fancyplain{}{\bfseries\thepage}}
\cfoot{}

% Interlinea
\linespread{1.3}

\begin{document}

%
% Dedica
%
\begin{titlepage}
  % No numero di pagina
  \thispagestyle{empty}
  \topmargin=6.5cm
  \raggedleft
  \large
  \em A Chiara\\
  A mio fratello
\end{titlepage}

\pagenumbering{roman}

\chapter*{Introduzione}
\rhead[\fancyplain{}{\bfseries
INTRODUZIONE}]{\fancyplain{}{\bfseries\thepage}}
\lhead[\fancyplain{}{\bfseries\thepage}]{\fancyplain{}{\bfseries
INTRODUZIONE}}
\addcontentsline{toc}{chapter}{Introduzione}
VoIP funziona su Internet: audio, video, testo e mail tutto in uno a

prezzo della connessione o poco più.

+ più potente della telefonia fissa
+ flessibile
+ individuale

- mobile della telefonia mobile.

Cellulari possono cambiare ripetitore al volo, i ripetitori sono distribuiti
su gran parte del territorio.

Connessione a Internet è legata all'indirizzo IP assegnato dal provider.
Un dispositivo Wi-Fi è confinato nell'area di copertura della rete a cui è
connesso. Cambiare rete significa cambiare indirizzo IP e quindi
interrompere la connessione. Una chiamata VoIP si interrope!

Scenario futuro in cui le città sono coperte da tante reti wireless con
diversi proprietari. Spostarsi nella città = cambiare spesso connessione
wireless = comunicazione interrotta di continuo.

Pausa tra disconnessione e riconnessione risolta con dispositivo
multi-homed. Serve middleware che nasconda le più interfacce wireless in una
sola (ULB)

Cambiamento di indirizzo IP risolto con Proxy Server.

Connessione wifi anello debole del percorso = serve meccanismo di
ritrasmissione veloce in modo da non avere ritardi audio/video (TED)

% Indice
\tableofcontents
\rhead[\fancyplain{}{\bfseries\leftmark}]{\fancyplain{}{\bfseries\thepage}}
\lhead[\fancyplain{}{\bfseries\thepage}]{\fancyplain{}{\bfseries
INDICE}}

% Capitoli
\chapter{Scenario}
\lhead[\fancyplain{}{\bfseries\thepage}]{\fancyplain{}{\bfseries\rightmark}}
\pagenumbering{arabic}

\begin{verbatim}
           +-----------------+                   +-------+
SOFTPHONE  |          iface0 +----AP0------------+       |
 |         |                 |                   |       |
 |         |          iface1 +----AP1------------+       |
 +---lo----+ LOAD            |        INTERNET   | PROXY +-...
           | BALANCER   ...  | ................. |       |
 +-MSG_ERR-+                 |                   |       |
 |         |          ifacen +----APn------------+       |
KERNEL     +-------------+---+                   +-------+
                         |
                    INTERFACE
                     MANAGER
\end{verbatim}

\chapter{Dettagli tecnici}
Come funziona il meccanismo del TED, dell'IM, etc. Riferimento
articolo dott. Ghini. (10/15 pp)

\chapter{Progettazione}
% Obiettivi a cui punto. Spiegazione soluzioni, algoritmi e scelte
% effettuate.
ULB sede dell'algoritmo di QoS.

Necessario confrontare diversi algoritmi, simulando il sistema perchè
impossibile sperimentare nella realtà in modo soddisfacente.

No puntare al realismo: per quello c'e' NS2.

Parametro riassuntivo della qualità di un collegamento: percentuale di
pacchetti persi. Questo parametro comprende tutte le cause che possono
portare alla perdita di pacchetti: interferenze, collisioni, distanza
dall'access point, etc.

\section{Obiettivi simulatore}
Deve poter gestire eventi, permettere di descrivere scenari in modo
semplice, permettere di sostituire diversi algoritmi nella simulazione
dell'ULB, dare in output una descrizione dettagliata e formattata di
tutto ciò che ha simulato. Questo output deve poter essere analizzato
con l'aiuto di utility come awk, perl, gnuplot.

\section{Obiettivi simulazione}
Confrontare i risultati:

\begin{itemize}
\item stesso scenario, algoritmi ULB differenti
\item stesso scenario, stesso algoritmo, parametri algoritmo differenti
\item scenari diversi per lo stesso algoritmo
\end{itemize}

Da tutto questo tirare fuori l'algoritmo migliore.

\section{Problemi}
Questa sezione illustra i problemi che devono essere affrontati dagli
algoritmi implementati nell'ULB. La sezione successiva mostrerà alcune
soluzioni e le confronterà tra loro.

\subsection{Ritrasmissione in tempi brevi}
ULB deve attendere le notifiche del TED e ritrasmettere i datagram che
sono stati segnalati come persi, ma senza superare i 150ms di ritardo.

\subsection{Rilevazione delle capacità del firmware}
I firmware delle schede di rete wireless sono in grado di notificare
sia l'avvenuta trasmissione di un datagram (ACK), sia la mancata
trasmissione (NAK). Putroppo non esiste uniformità di comportamento
tra i vari modelli sul mercato, ed è possibile che il firmware di una
scheda sia in grado di notificare solo ACK, oppure solo NAK, oppure
entrambi.

Mancando un meccanismo nel kernel Linux per conoscere con certezza le
capacità di ogni firmware, ULB deve dedurle dal comportamento
osservato.

\subsubsection{Firmware solo ACK}


\subsubsection{Firmware solo NAK}

\chapter{simulatore}
Implementazione del simulatore.

\chapter{Valutazioni}
Metriche con cui valuto il mio sistema. Deduzioni su misurazioni
effettuate. Perchè ho misurato certe cose e non altre.

\chapter*{Conclusioni}
\rhead[\fancyplain{}{\bfseries
CONCLUSIONI}]{\fancyplain{}{\bfseries\thepage}}
\lhead[\fancyplain{}{\bfseries\thepage}]{\fancyplain{}{\bfseries
CONCLUSIONI}}
\addcontentsline{toc}{chapter}{Conclusioni}
Un temerario che scriva una patch per ns2.

\end {document}
