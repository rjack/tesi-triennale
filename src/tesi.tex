% vim: tabstop=2
\documentclass[12pt,a4paper,openright,twoside]{book}
\usepackage[italian]{babel}
\usepackage[utf8]{inputenc}
\usepackage{fancyhdr}
\usepackage{indentfirst}
\usepackage{graphicx}

\oddsidemargin=30pt
\evensidemargin=20pt

% Sillabazione
\hyphenation{}

% Interlinea
\linespread{1.3}

\begin{document}

%
% Dedica
%
\begin{titlepage}
	% No numero di pagina
	\thispagestyle{empty}
	\topmargin=6.5cm
	\raggedleft
	\large
	\em
	A Chiara\\
	Al fratello
\end{titlepage}

\pagenumbering{roman}

\chapter*{Introduzione}
\addcontentsline{toc}{chapter}{Introduzione}

	Di tre pagine, da fare per ultima insieme alle conclusioni, così per passare
	il tempo.

\tableofcontents

\chapter{Scenario}
\pagenumbering{arabic}

	Presentazione dello scenario del problema, degli obiettivi (5/10 pp).

\chapter{Dettagli tecnici}

	Come funziona il meccanismo del TED, dell'IM, etc. Riferimento articolo
	dott. Ghini. (10/15 pp)

\chapter{Progettazione}

	Obiettivi a cui punto. Spiegazione soluzioni, algoritmi e scelte effettuate.

\chapter{simulatore}

	Implementazione del simulatore.

\chapter{Valutazioni}

	Metriche con cui valuto il mio sistema.\\
	Deduzioni su misurazioni effettuate.\\
	Perchè ho misurato certe cose e non altre.

\chapter{Conclusioni e sviluppi futuri}

	Un temerario che scriva una patch per ns2.

\end {document}
