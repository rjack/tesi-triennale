\documentclass[12pt,a4paper,openright,twoside]{book}
\usepackage[italian]{babel}
\usepackage[utf8]{inputenc}
\usepackage{fancyhdr}
\usepackage{indentfirst}
\usepackage{graphicx}

\pagestyle{fancy}

\oddsidemargin=30pt
\evensidemargin=20pt

% Sillabazione
\hyphenation{}

% Intestazione e piè di pagina
\pagestyle{fancy}\addtolength{\headwidth}{20pt}
\renewcommand{\chaptermark}[1]{\markboth{\thechapter.\ #1}{}}
\renewcommand{\sectionmark}[1]{\markright{\thesection \ #1}{}}
\rhead[\fancyplain{}{\bfseries\leftmark}]{\fancyplain{}{\bfseries\thepage}}
\cfoot{}

% Interlinea
\linespread{1.3}

\begin{document}

%
% Dedica
%
\begin{titlepage}
  % No numero di pagina
  \thispagestyle{empty}
  \topmargin=6.5cm
  \raggedleft
  \large
  \em A Chiara\\
  A mio fratello
\end{titlepage}

\pagenumbering{roman}

\chapter*{Introduzione}
\rhead[\fancyplain{}{\bfseries
INTRODUZIONE}]{\fancyplain{}{\bfseries\thepage}}
\lhead[\fancyplain{}{\bfseries\thepage}]{\fancyplain{}{\bfseries
INTRODUZIONE}}
\addcontentsline{toc}{chapter}{Introduzione}
VoIP funziona su Internet: audio, video, testo e mail tutto in uno a
prezzo della connessione o poco più.

+ più potente della telefonia fissa
+ flessibile
+ individuale

- mobile della telefonia mobile.

Cellulari possono cambiare ripetitore al volo, i ripetitori sono distribuiti
su gran parte del territorio.

Connessione a Internet è legata all'indirizzo IP assegnato dal provider.
Un dispositivo Wi-Fi è confinato nell'area di copertura della rete a cui è
connesso. Cambiare rete significa cambiare indirizzo IP e quindi
interrompere la connessione. Una chiamata VoIP si interrope!

Scenario futuro in cui le città sono coperte da tante reti wireless con
diversi proprietari. Spostarsi nella città = cambiare spesso connessione
wireless = comunicazione interrotta di continuo.

Pausa tra disconnessione e riconnessione risolta con dispositivo
multi-homed. Serve middleware che nasconda le più interfacce wireless in una
sola (ULB)

Cambiamento di indirizzo IP risolto con Proxy Server.

Connessione wifi anello debole del percorso = serve meccanismo di
ritrasmissione veloce in modo da non avere ritardi audio/video (TED)

% Indice
\tableofcontents
\rhead[\fancyplain{}{\bfseries\leftmark}]{\fancyplain{}{\bfseries\thepage}}
\lhead[\fancyplain{}{\bfseries\thepage}]{\fancyplain{}{\bfseries
INDICE}}

% Capitoli
\chapter{Scenario}
\lhead[\fancyplain{}{\bfseries\thepage}]{\fancyplain{}{\bfseries\rightmark}}
\pagenumbering{arabic}

Lo scenario da considerare è illustrato in figura
[TODO-FIGURASCENARIO] e descritto in [TODO-REFPAPERGHINI]. Si assuma
che esista una comunicazione vocale tra due sistemi A e B. Il sistema
A è un dispositivo mobile dotato di più di un'interfaccia wireless,
ognuna delle quali è connessa a un access point. I vari access point
possono appartenere a infrastrutture di rete e domini differenti e
quindi essere completamente indipendenti l'uno dall'altro. Gli access
point sono connessi a Internet attraverso una connessione via cavo. Il
sistema B è un dispositivo fisso, connesso a Internet via cavo.

La comunicazione VoIP tra i sistemi A e B si affida quindi a due tipi
di connessione: quella wireless, tra A e gli access point, e quella
via cavo tra gli access point e B; di queste la prima la più
problematica, perchè soffre dei seguenti inconvenienti:

\begin{enumerate}
\item alta latenza
\item frequenti errori di trasmissione
\item disconnessione e riconnessione nel cambiare access point.
\end{enumerate}

I primi due problemi possono degradare la qualità di una comunicazione
VoIP via wireless fino a renderla impossibile; il terzo impedisce la
mobilità del sistema obbligandolo a rimanere connesso sempre alla
stessa infrastruttura wireless, pena disconnessione e conseguente
interruzione della comunicazione.

Per avere una buona qualità del servizio, TODO-REFPAPERGHINI impone i
seguenti vincoli:
\begin{enumerate}
\item latenza da sistema a sistema sotto i 150ms
\item percentuale di pacchetti persi sotto il 10\%.
\item riconnessione trasparente in caso di cambio di access point.
\end{enumerate}

Il sistema RWMA descritto in TODO-REFPAPERGHINI è costituito da un
insieme di componenti che mira a risolvere o mitigare i problemi
detti:

\begin{itemize}
\item Transmission Error Detection.
\item UDP Load Balancer
\item Monitor
\item Proxy Server
\end{itemize}

CONTINUA QUI

\begin{verbatim}
           +-----------------+                   +-------+
SOFTPHONE  |          iface0 +----AP0------------+       |
 |         |                 |                   |       |
 |         |          iface1 +----AP1------------+       |
 +---lo----+ LOAD            |        INTERNET   | PROXY +-...
           | BALANCER   ...  | ................. |       |
 +-MSG_ERR-+                 |                   |       |
 |         |          ifacen +----APn------------+       |
KERNEL     +-------------+---+                   +-------+
                         |
                    INTERFACE
                     MANAGER
\end{verbatim}

\chapter{Dettagli tecnici}


\chapter{Simulazione}
% Obiettivi a cui punto. Spiegazione soluzioni, algoritmi e scelte
% effettuate.
ULB e Proxy Server sono i componenti del sistema che eseguono le
scelte necessarie a garantire una migliore qualità del
servizio. Mentre i componenti Interface Manager, Linux Kernel,
Softphone, Wireless Interface e Access Point hanno comportamenti
predefiniti e sostanzialmente non modificabili, ULB e PS possono
implementare  diverse tecniche di analisi della qualità delle
connessioni e varie strategie di recupero degli errori.

Compito del simulatore è fornire un ambiente virtuale in cui questi
diversi algoritmi possano essere eseguiti, analizzati e confrontati.

È importante notare che, a questo stadio di sviluppo, il simulatore
non punta al realismo della simulazione. Per esempio, lo scenario
wireless è di molto semplificato, non venendo presi in considerazione
parametri fisici come la qualità del segnale radio in relazione alla
distanza di un client dall'access point. Tutti i dettagli di un link
wireless sono astratti dietro ai parametri di banda, latenza e
probabilità di errore; in particolare la rilevazione dell'ultimo
parametro sarà fondamentale per determinare la scelta dell'interfaccia
migliore. In quest'ottica, per simulare un calo di qualità del segnale
wireless dovuto all'aumentare della distanza tra client e access point
è sufficiente decidere quanto questo evento incida sulla probabilità
d'errore del collegamento e modificarla di conseguenza.

\section{Obiettivi simulatore}
Il simulatore è stato progettato secondo i principi di
\begin{itemize}
\item modularità
\item componibilità
\item generalità
\end{itemize}

\subsection{Modularità}
Modularità significa che i simulatori possono essere combinati tra di
loro per rappresentare nuovi scenari con componenti esistenti; per
esempio un'oggetto simulante un'interfaccia wireless in modo
soddisfacente, è autonomo e riutilizzabile in ogni scenario che
necessiti di un'interfaccia wireless. Questo permette il riuso del
codice in modo estremamente diretto.

\subsection{Componibilità}
Componibilità significa che ogni simulatore che partecipa in una
simulazione può essere a sua volta composto da più simulatori che
costituiscono le sue parti interne; questo permette di mantenere
gestibile la complessità della simulazione, senza dover sacrificare il
livello di dettaglio voluto.

\subsection{Generalità}
Il simulatore vuole essere un framework il più possibile
generale. L'obiettivo è essere utilizzabile in qualsiasi contesto in
cui il problema sia definibile in termini di eventi discreti.


\section{Obiettivi simulazione}
Confrontare i risultati:
\begin{itemize}
\item stesso scenario, algoritmi ULB differenti
\item stesso scenario, stesso algoritmo, parametri algoritmo differenti
\item scenari diversi per lo stesso algoritmo
\end{itemize}
Da tutto questo tirare fuori l'algoritmo migliore.

\section{Problemi}
Questa sezione illustra i problemi che devono essere affrontati dagli
algoritmi implementati nell'ULB. La sezione successiva mostrerà alcune
soluzioni e le confronterà tra loro.

\subsection{Ritrasmissione in tempi brevi}
ULB deve attendere le notifiche del TED e ritrasmettere i datagram che
sono stati segnalati come persi, ma senza superare i 150ms di ritardo.

\subsection{Valutazione interfacce}
ULB deve monitorare il comportamento delle varie interfacce dell'host
su cui viene eseguito e valutarne la qualità. L'interfaccia che
risulta essere la migliore viene utilizzata per l'invio dei dati. È
necessario definire sia quali siano i parametri secondo cui valutare
la qualità di un'interfaccia, sia le strategie di monitoraggio e
misurazione di questi parametri. La valutazione comprende sia
l'attività \emph{first hop} tra interfaccia e access point, sia quella
\emph{full path} tra interfaccia e Proxy Server.

\subsection{Rilevazione delle capacità del firmware}
I firmware delle schede di rete wireless sono in grado di notificare
sia l'avvenuta trasmissione di un datagram (ACK), sia la mancata
trasmissione (NAK). Putroppo non esiste uniformità di comportamento
tra i vari modelli sul mercato, ed è possibile che il firmware di una
scheda sia in grado di notificare solo ACK, oppure solo NAK, oppure
entrambi. Mancando un meccanismo nel kernel Linux per conoscere con
certezza le capacità di ogni firmware, ULB deve dedurle dal
comportamento osservato.

\subsubsection{Il problema del firmware silenzioso}
Il tipo di firmware si deduce dal tipo di notifiche ricevute dallo
stesso ed è quindi un meccanismo banale. Esistono però due casi in cui
il firmware può rimanere ``silenzioso'' e quindi impossibile da
rilevare:
\begin{enumerate}
\item quando una connessione pessima, che perde tutti i pacchetti,
  viene gestita da un firmware ACK che notifica solo successi, oppure
\item quando una connessione ottima, che non perde nessun pacchetto,
  viene gestita da un firmware NAK che notifica solo fallimenti.
\end{enumerate}
Non esiste un comportamento predefinito che soddisfi entrambe le
situazioni: la strategia del ritrasmettere ogni datagram che non
riceva notifiche entro un certo timeout funziona solo nella prima
situazione e risulta completamente inadeguato nella seconda.

\subsection{Conversazione unidirezionale}
Poichè l'host su cui risiede il softphone è dotato di più interfacce,
il Proxy Server si trova nella necessità di scegliere a quale tra
queste debba inviare i dati. Normalmente la scelta cade
sull'interfaccia che ha spedito gli ultimi dati ricevuti: se ULB l'ha
scelta come interfaccia migliore, il Proxy Server può fidarsi. Può
però accadere che il softphone non invii alcun dato perchè l'utente si
limita ad ascoltare; in questo caso il Proxy Server non ha cognizione
di quale sia l'interfaccia a cui inviare i dati.

\section{Soluzioni}
Tutti gli algoritmi sono stati progettati in modo da essere il più
possibile economici in termini di banda, tempi e energia.

Economia di banda significa limitare al minimo indispensabile
l'overhead nelle trasmissioni. In un contesto \emph{multi-homed} dove
ogni host possiede mediamente dalle due alle quattro interfacce,
ognuna di queste genera il proprio overhead che si va a sommare a
quello generato dalle altre.

L'economia di tempi è richiesta dalla caratterizzazione \emph{soft
  real-time} del problema, per cui un risultato ottenuto in ritardo
diventa inutilizzabile o fuorviante.

L'economia di energia dipende dalla natura dei client, che sono
tipicamente dispositivi mobili alimentati a batteria. Le interfacce
wireless sono periferiche avide di consumi e moltiplicandone il numero
si rischia di abbattere l'autonomia del client. Ancora una volta,
limitare l'overhead è l'unica via.

\subsection{Ritrasmissione in tempi brevi}
Fissato il ritardo massimo a 150ms, ULB associa un timeout di tale
durata ad ogni datagram ricevuto dal softphone. Scaduto il timeout il
datagram viene scartato.

\subsection{Valutazione interfacce}
Si è scelto di suddividere la valutazione di ogni interfaccia WiFi in
due sottovalutazioni differenti: una che considera il comportamento
nel \emph{first hop} e un'altra che considera quello nel \emph{full
  path}.

\subsubsection{First hop}
La valutazione \emph{first hop} rappresenta la qualità del
collegamento wireless tra interfaccia e access point, ed è una
funzione delle notifiche TED.
% TODO formula.

\subsubsection{Full path}
La valutazione \emph{full path} rappresenta l'affidabilità di tutto il
percorso dall'interfaccia al proxy server e ritorno. Ogni interfaccia
spedisce ogni 500ms un datagram che viene rispedito indietro dal Proxy
Server, come un semplice ping.

\chapter{simulatore}
Implementazione del simulatore.

\chapter{Valutazioni}
Metriche con cui valuto il mio sistema. Deduzioni su misurazioni
effettuate. Perchè ho misurato certe cose e non altre.

\chapter*{Conclusioni}
\rhead[\fancyplain{}{\bfseries
CONCLUSIONI}]{\fancyplain{}{\bfseries\thepage}}
\lhead[\fancyplain{}{\bfseries\thepage}]{\fancyplain{}{\bfseries
CONCLUSIONI}}
\addcontentsline{toc}{chapter}{Conclusioni}
Un temerario che scriva una patch per ns2.

\end {document}
